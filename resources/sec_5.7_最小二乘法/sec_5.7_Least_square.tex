%\documentclass[handout]{beamer}
\documentclass[11pt]{beamer}
 % 1. packages

 % ----------- fonts and symbles ---------
\usepackage{amsmath,amssymb,amsfonts,amsthm}
%\usepackage{CJK}
\usepackage{dsfont}
\usepackage{mathrsfs}
\usepackage{eucal} % for \mathcal

%\renewcommand{\rmdefault}{ptm}


%\usepackage{fontspec}
%\newfontfamily\monaco{Monaco}

%\usepackage{mathbbold} %,bbold

 \usepackage{textcomp} % for \textnormal{\textperthousand}
% -----------------





%\usepackage{slashbox}
%\usepackage[margin=2.2cm]{geometry} % |geometry| package clash with |booktabs| package
%\usepackage{cases}
% -------- tables -------
\usepackage{booktabs} % for \toprule, \bottomrule
\usepackage{tabularx}
\usepackage{multirow}
% --------- figures ---------
\usepackage{graphicx}
% ---------- algorithms -------
\usepackage{algorithm}
\usepackage{algorithmic}
%\usepackage{footnote}
    % |footnote| package occurs error:
    % Runaway argument?
    % \def \insertfootnotetext {\@@ }\def \insertfootnotemark {\@makefnmark \ETC.

\usepackage{listings}

\usepackage[linewidth=1pt]{mdframed} % for  mdframe environment




 \usepackage{color}
 \usepackage{xcolor}     %¸ßÁÁʹÓõÄÑÕÉ«

\usepackage{setspace}
%%\usepackage{type1cm}
\usepackage{adjustbox} % for \adjustbox

\usepackage{accsupp}
\newcommand{\emptyaccsupp}[1]{\BeginAccSupp{ActualText={}}#1\EndAccSupp{}}




%%   figures and tables
\graphicspath{{figure/}}


% 2. new commands

% 2.0 common commands
%\newcommand{\bc}{\begin{center}}
%\newcommand{\ec}{\end{center}}
%\newcommand{\ba}{\begin{array}}
%\newcommand{\ea}{\end{array}}
%\newcommand{\be}{\begin{equation}}
%\newcommand{\ee}{\end{equation}}

% 2.1 colors
\definecolor{dgrey}{rgb}{0.30,0.30,0.30}
\definecolor{lred}{rgb}{0.50,0.00,0.50}
\definecolor{lblue}{rgb}{0.8,0.8,1}
\definecolor{dred}{rgb}{0.6,0,0}
\definecolor{dblue}{rgb}{0,0,0.5}
\definecolor{dgrey}{rgb}{0.35,0.35,0.35}
\definecolor{rred}{rgb}{0.9,0,0}
\definecolor{mylblue}{rgb}{0.3,0.2, 0.8}

\definecolor{commentcolor}{RGB}{85,139,78}
\definecolor{stringcolor}{RGB}{206,145,108}
\definecolor{keywordcolor}{RGB}{34,34,250}
\definecolor{backcolor}{RGB}{220,220,220}

\newcommand{\blue}[1]{{\color{blue}#1}}
\newcommand{\dblue}[1]{{\color{dblue}#1} }
\newcommand{\red}[1]{{\color{red}#1}}
\newcommand{\dred}[1]{{\color{dred}#1}}
\newcommand{\cyan}[1]{{\color{cyan}#1}}
\newcommand{\bfblue}[1]{\textbf{\color{dblue}#1} }
\newcommand{\bfred}[1]{\textbf{\color{dred}#1} }
\newcommand{\green}[1]{{\color{green}#1}}
%\newcommand{\alert}[1]{{\color{red}#1}}
\newcommand{\black}[1]{{\color{black}#1}}
\newcommand{\light}[1]{{\color{blue}\textbf{#1}}}
\newcommand{\hot}[1]{{\color{dred}#1}}
 \newcommand{\highlight}[1]{ \textbf{\color{mylblue}#1}}
 \newcommand{\important}[1]{{\color{red}#1}} % for highlighting  some words

 \newcommand{\mystar}{\dred{$^{\clubsuit}$ }}
  \newcommand{\doublestar}{\dred{$^{\clubsuit\clubsuit}$ }}

\newcommand{\mynote}[1]{{\footnotesize \color{mylblue}#1}}

 \newcommand{\hint}[1]{{\small \color{mylblue}#1}}
\newcommand{\smallhint}[1]{{\small \color{dgrey}#1}}
\newcommand{\footnotehint}[1]{{\footnotesize \color{dgrey}#1}}
\newcommand{\tinyhint}[1]{{\tiny \color{dgrey}#1}}
\newcommand{\mytitle}[1]{\medskip{\large \textbf{\color{mylblue}#1}}}
\newcommand{\normaltitle}[1]{\medskip{ \textbf{\color{mylblue}#1}}}

%\newcommand{\head}[1]{\textbf{\large\color{blue}#1}}
%\newcommand{\heading}[1]{\textbf{\large\color{blue}#1}}

\newcommand{\myfbox}[2]{ \bigskip \begin{center} \fbox{\parbox{#1}{ #2  }} \end{center}\bigskip }

\newcommand{\myvar}[1]{}
%\newcommand{\mynote}[1]{#1}

% 2.2 mathematical symbols

\newcommand{\drightarrow}{\stackrel{d.}{\rightarrow}}
\newcommand{\prightarrow}{\stackrel{p.}{\rightarrow}}
\newcommand{\bernoulli}{\textnormal{Ber}}
\newcommand{\cov}{\mathsf{Cov}}
\newcommand{\corr}{\mathbf{Corr}}
\newcommand{\regret}{\textnormal{Regret}}
\newcommand{\conv}{\textnormal{conv}}
\newcommand{\dotdiv}{\stackrel{\centerdot}{-}}
\newcommand{\dom}{\textnormal{dom}}
\newcommand{\convergenceinprob}{\stackrel{P}{\rightarrow}}
\newcommand{\convergenceindist}{\rightsquigarrow}
\newcommand{\probability}{\mathbb{P}}
\newcommand{\expectation}{\mathbb{E}}
\newcommand{\epi}{\textnormal{epi}}
\newcommand{\variance}{\mathbb{V}}
\newcommand{\var}[1]{\mathbb{V}(#1)}
\newcommand{\covariance}{\mathsf{Cov}}
\newcommand{\empiricalrisk}[1]{\hat{R}(#1)}
\newcommand{\expectedrisk}[1]{R(#1)}
\newcommand{\mgf}[1]{\psi_{#1}(\lambda)}
\newcommand{\mgfexpansion}[1]{\expectation[e^{\lambda#1}]}
\newcommand{\mgfmultivariate}[1]{\expectation[e^{\lambda^\transpose#1}]}
\newcommand{\transpose}{{\mathsf{T}}}
\newcommand{\real}{\mathbb{R}}
\newcommand{\gaussian}[2]{\mathcal{N}(#1,#2)}
\newcommand{\subGaussian}[1]{\mathsf{subG}(#1)}
\newcommand{\indicator}[1]{\mathbb{I}[#1]}
\newcommand{\x}[1]{x^{(#1)}}
\newcommand{\y}[1]{y^{(#1)}}
\newcommand{\z}[1]{z^{(#1)}}
\newcommand{\feature}{x}
\newcommand{\response}{y}
\newcommand{\supofempiricalprocess}{\|\mathbb{P}_n-\mathbb{P}\|_{\decisionspace}}
\newcommand{\decisionspace}{\mathscr{F}}
\newcommand{\decisionfunction}{f}
\newcommand{\featurespace}{\mathcal{X}}
\newcommand{\classifierestimate}{\widehat{h}}
\newcommand{\classifiertrue}{h^\star}
\newcommand{\classifier}{h}
\newcommand{\hypothesisclass}{\mathcal{H}}
\newcommand{\dataset}{\mathcal{D}}
\newcommand{\defineas}{\stackrel{\textnormal{def}}{=}}
\newcommand{\rademachercomplexity}[1]{\mathsf{Rad}_n\left(#1\right)}
\newcommand{\loss}{\ell}
\newcommand{\composite}{\circ}
\newcommand{\convexhull}{\mathsf{conv}}
\newcommand{\norm}[2][2]{\|#2\|_{#1}}
\newcommand{\shatteringcoefficient}[2]{\mathcal{S}(#1,#2)}
\newcommand{\vcdimension}[1]{\mathsf{VC}\left(#1\right)}
\newcommand{\rank}{\mathsf{rank}}
\newcommand{\innerproduct}[2]{\left\langle #1, #2\right\rangle}
\newcommand{\modelparameter}{\theta}
\newcommand{\ball}[3][]{\mathcal{B}_{{#1}}\left(#2,#3\right)}
\newcommand{\metric}{d}
\newcommand{\coveringnumber}[4][]{N_{{#1}}\left(#2,#3,#4\right)}
\newcommand{\trace}{\textnormal{tr}}
\newcommand{\std}{\textnormal{std}}
\newcommand{\sgn}{\textnormal{sign}}
%\renewcommand{\span}{\textnormal{span}}

 % do not overwrite the existing command \span
 % as it leads to an error of
 %  "Missing # Inserted in Alignment Preamble" for ``align'' environment

\newcommand{\myspan}{\textnormal{span}}

%%%
\newcommand{\rightarrowd}{\stackrel{d}{\rightarrow}}
\newcommand{\rightarrowp}{\stackrel{p}{\rightarrow}}
\newcommand{\defeq}{ \stackrel{\textnormal{def}}{=}}
\newcommand{\proj}{ \textnormal{Proj}}
\newcommand{\dist}{\textnormal{dist}}

\newcommand{\argmax}{\textnormal{argmax}}
\newcommand{\argmin}{\textnormal{argmin}}
\newcommand{\subg}{\textnormal{subG}}


 \newcommand{\bba}{\mathbb{A}}
\newcommand{\bbb}{\mathbb{B}}
\newcommand{\bbc}{\mathbb{C}}
\newcommand{\bbd}{\mathbb{D}}
\newcommand{\bbe}{\mathbb{E}}
\newcommand{\bbf}{\mathbb{F}}
\newcommand{\bbg}{\mathbb{G}}
\newcommand{\bbh}{\mathbb{H}}
\newcommand{\bbi}{\mathbb{I}}
\newcommand{\bbj}{\mathbb{J}}
\newcommand{\bbk}{\mathbb{K}}
\newcommand{\bbl}{\mathbb{L}}
\newcommand{\bbm}{\mathbb{M}}
\newcommand{\bbn}{\mathbb{N}}
\newcommand{\bbo}{\mathbb{O}}
\newcommand{\bbp}{\mathbb{P}}
\newcommand{\bbq}{\mathbb{Q}}
\newcommand{\bbr}{\mathbb{R}}
\newcommand{\bbs}{\mathbb{S}}
\newcommand{\bbt}{\mathbb{T}}
\newcommand{\bbu}{\mathbb{U}}
\newcommand{\bbv}{\mathbb{V}}
\newcommand{\bbw}{\mathbb{W}}
\newcommand{\bbx}{\mathbb{X}}
\newcommand{\bby}{\mathbb{Y}}
\newcommand{\bbz}{\mathbb{Z}}

\newcommand{\bfa}{\mathbf{a}}
\newcommand{\bfb}{\mathbf{b}}
\newcommand{\bfc}{\mathbf{c}}
\newcommand{\bfd}{\mathbf{d}}
\newcommand{\bfe}{\mathbf{e}}
\newcommand{\bff}{\mathbf{f}}
\newcommand{\bfg}{\mathbf{g}}
\newcommand{\bfh}{\mathbf{h}}
\newcommand{\bfi}{\mathbf{i}}
\newcommand{\bfj}{\mathbf{j}}
\newcommand{\bfk}{\mathbf{k}}
\newcommand{\bfl}{\mathbf{l}}
\newcommand{\bfm}{\mathbf{m}}
\newcommand{\bfn}{\mathbf{n}}
\newcommand{\bfo}{\mathbf{o}}
\newcommand{\bfp}{\mathbf{p}}
\newcommand{\bfq}{\mathbf{q}}
\newcommand{\bfr}{\mathbf{r}}
\newcommand{\bfs}{\mathbf{s}}
\newcommand{\bft}{\mathbf{t}}
\newcommand{\bfu}{\mathbf{u}}
\newcommand{\bfv}{\mathbf{v}}
\newcommand{\bfw}{\mathbf{w}}
\newcommand{\bfx}{\mathbf{x}}
\newcommand{\bfy}{\mathbf{y}}
\newcommand{\bfz}{\mathbf{z}}

\newcommand{\bfA}{\mathbf{A}}
\newcommand{\bfB}{\mathbf{B}}
\newcommand{\bfC}{\mathbf{C}}
\newcommand{\bfD}{\mathbf{D}}
\newcommand{\bfE}{\mathbf{E}}
\newcommand{\bfF}{\mathbf{F}}
\newcommand{\bfG}{\mathbf{G}}
\newcommand{\bfH}{\mathbf{H}}
\newcommand{\bfI}{\mathbf{I}}
\newcommand{\bfJ}{\mathbf{J}}
\newcommand{\bfK}{\mathbf{K}}
\newcommand{\bfL}{\mathbf{L}}
\newcommand{\bfM}{\mathbf{M}}
\newcommand{\bfN}{\mathbf{N}}
\newcommand{\bfO}{\mathbf{O}}
\newcommand{\bfP}{\mathbf{P}}
\newcommand{\bfQ}{\mathbf{Q}}
\newcommand{\bfR}{\mathbf{R}}
\newcommand{\bfS}{\mathbf{S}}
\newcommand{\bfT}{\mathbf{T}}
\newcommand{\bfU}{\mathbf{U}}
\newcommand{\bfV}{\mathbf{V}}
\newcommand{\bfW}{\mathbf{W}}
\newcommand{\bfX}{\mathbf{X}}
\newcommand{\bfY}{\mathbf{Y}}
\newcommand{\bfZ}{\mathbf{Z}}


\newcommand{\bfSigma}{\mathbf{\Sigma}}
\newcommand{\bfrho}{\mathbf{\rho}}

\newcommand{\cala}{\mathcal{A}}
\newcommand{\calb}{\mathcal{B}}
\newcommand{\calc}{\mathcal{C}}
\newcommand{\cald}{\mathcal{D}}
\newcommand{\cale}{\mathcal{E}}
\newcommand{\calf}{\mathcal{F}}
\newcommand{\calg}{\mathcal{G}}
\newcommand{\calh}{\mathcal{H}}
\newcommand{\cali}{\mathcal{I}}
\newcommand{\calj}{\mathcal{J}}
\newcommand{\calk}{\mathcal{K}}
\newcommand{\call}{\mathcal{L}}
\newcommand{\calm}{\mathcal{M}}
\newcommand{\caln}{\mathcal{N}}
\newcommand{\calo}{\mathcal{O}}
\newcommand{\calp}{\mathcal{P}}
\newcommand{\calq}{\mathcal{Q}}
\newcommand{\calr}{\mathcal{R}}
\newcommand{\cals}{\mathcal{S}}
\newcommand{\calt}{\mathcal{T}}
\newcommand{\calu}{\mathcal{U}}
\newcommand{\calv}{\mathcal{V}}
\newcommand{\calw}{\mathcal{W}}
\newcommand{\calx}{\mathcal{X}}
\newcommand{\caly}{\mathcal{Y}}
\newcommand{\calz}{\mathcal{Z}}


% 3. theorem and environments

%\newtheorem{theorem}{Theorem}%[section]
\newtheorem{proposition}{Proposition}%[section]
%\newtheorem{property}{Property}%[section]
%\newtheorem{lemma}{Lemma}%[section]
%\newtheorem{corollary}{Corollary}%[section]
%\newtheorem{definition}{Definition}%[section]
%\newtheorem{example}{Example}%[section]
%\newtheorem{remark}{Remark}%[section]
%\newtheorem{note}{Note}%[section]
%\newtheorem{problem}{Problem}%[section]
\newtheorem{exercise}{Exercise}
%\newtheorem{assumption}{Assumption}
\newtheorem*{lemma_star}{Lemma}
\newtheorem*{theorem_star}{Theorem}

%\newenvironment{summary}[1][Summary]{\par\medskip   \color{dred}\textbf{\large#1. } }{ \medskip}
%\newenvironment{remark}[1][Remark]{\par\medskip  \begin{small} \color{dblue}\textbf{#1. } }{ \end{small}\medskip}
%\renewenvironment{proof}[1][Proof]{\noindent\textbf{#1.} }{\mbox{} \hfill{\small\textrm{$\Box$}}\vspace{1ex}}
% \newenvironment{answer}[1][Answer]{\par\medskip \color{dblue}\textbf{\large#1. }}{ \medskip}

\newenvironment{summary}[1][总结]{\par\medskip   \color{dred}\textbf{\large#1 } }{ \medskip}
\newenvironment{remark}[1][注意]{\par\medskip   \color{dblue}\textbf{\large#1 } }{ \medskip}
\newenvironment{footnoteremark}{ \color{dblue}\begin{footnotesize} }{\end{footnotesize}}
\renewenvironment{proof}[1][证明]{\noindent\textbf{#1.} }{\mbox{} \hfill{\small\textrm{$\Box$}}\vspace{1ex}}
 \newenvironment{question}[1][Q.]{\par\medskip {\color{lred}\large#1}}{ \medskip}
 \newenvironment{answer}[1][Answer]{\par\medskip \color{dblue}\textbf{\large#1 }}{ \medskip}

% 4. beamer setting




%\newtheorem{definition}{\textbf{¶¨Òå}}[section]
%\newtheorem{proposition}[definition] { \textbf{ÃüÌâ}}
%\newtheorem{lemma}[definition] { \textbf{ÒýÀí}}
%\newtheorem{theorem}[definition]{ \textbf{¶¨Àí}}
%\newtheorem{corollary}[definition] { \textbf{ÍÆÂÛ}}
%\newtheorem{remark}[definition] { \textbf{×¢}}
%\newtheorem{example}[definition] { \textbf{Àý}}

%\newcommand{\shadow}[1]{\begin{center}
%\bf{\textcolor{dblue}{\shadowbox{\parbox{3.8in}
% {\textcolor{red}
% {\vspace{1mm}#1}}}}}
%\end{center}}
%
%\newcommand{\head}[1]{\begin{center}
%\bf{\textcolor{dblue}{\shadowbox{\parbox{3.8in}
% {\textcolor{dred}
% {\vspace{1mm}#1}}}}}
%\end{center}}
%
%
%\newcommand{\heading}[1]{%
%  \begin{center}
%    \large\bf
%    \shadowbox{#1}%
%  \end{center}
%\vspace{1ex minus 1ex}}

% set  space above and below math equations in display style

\expandafter\def\expandafter\normalsize\expandafter{%
    \normalsize
    \setlength\abovedisplayskip{1.5ex}
    \setlength\belowdisplayskip{1.2ex}
    \setlength\abovedisplayshortskip{0.5ex}
    \setlength\belowdisplayshortskip{0.5ex}
}

% Ìí¼ÓÒ³Âë´úÂ룬¹È¸èÕÒµ½µÄ¡£
\addtobeamertemplate{navigation symbols}{}{%
    %\usebeamerfont{footline}%
    %\usebeamercolor[fg]{footline}%
    \setbeamercolor{footline}{fg=blue}
    \setbeamerfont{footline}{series=\bfseries}
    \hspace{1em}%
    \normalsize{\insertframenumber/\inserttotalframenumber}
}

% section numbering
\setbeamertemplate{section in toc}[sections numbered]
\setbeamertemplate{subsection in toc}[subsections numbered]



\lstset{                        %¸ßÁÁ´úÂëÉèÖÃ
%basicstyle=\small, % print whole listing small
%basicstyle=\footnotesize\sffamily, % print whole listing small
basicstyle=\footnotesize\rmfamily, % print whole listing small
%basicstyle=\rmfamily, % print whole listing small
    language=python,                    %PythonÓï·¨¸ßÁÁ
    %linewidth=0.9\linewidth,            %Áбílist¿í¶È
    %basicstyle=\ttfamily,              %ttÎÞ·¨ÏÔʾ¿Õ¸ñ
    commentstyle=\color{commentcolor},  %×¢ÊÍÑÕÉ«
    keywordstyle=\color{keywordcolor},  %¹Ø¼ü´ÊÑÕÉ«
    stringstyle=\color{stringcolor},    %×Ö·û´®ÑÕÉ«
    %showspaces=true,                   %ÏÔʾ¿Õ¸ñ
    numbers=left,                       %ÐÐÊýÏÔʾÔÚ×ó²à
    %numberstyle=\tiny\emptyaccsupp,     %ÐÐÊýÊý×Ö¸ñʽ
    numberstyle=\tiny,                  %ÐÐÊýÊý×Ö¸ñʽ
    numbersep=5pt,                      %Êý×Ö¼ä¸ô
    frame=single,                       %¼Ó¿ò
    framerule=0pt,                      %²»»®Ïß
    %escapeinside=@@,                    %ÌÓÒݱêÖ¾
    escapeinside=``,                    %ÌÓÒݱêÖ¾
    emptylines=1,                       %
    xleftmargin=3em,                    %list×ó±ß¾à
    backgroundcolor=\color{backcolor},  %ÁÐ±í±³¾°É«
    tabsize=4,                          %ÖƱí·û³¤¶ÈΪ4¸ö×Ö·û
    %gobble=4                            %ºöÂÔÿÐдúÂëÇ°4¸ö×Ö·û
    breaklines=true,
    extendedchars=false
    }

\lstdefinestyle{numbers}{numbers=left, stepnumber=1, numberstyle=\tiny, numbersep=10pt}
 \lstdefinestyle{nonumbers}{numbers=none}

\newcommand{\alertcode}[1]{{\color{red}#1}} % used for alerting codes

%\lstset{numbers=left, numberstyle=\tiny,
%keywordstyle=\color{blue!70},
%commentstyle=\color{red!50!green!50!blue!50},
%frame=shadowbox,
%rulesepcolor=\color{red!20!green!20!blue!20},
%escapeinside=``,
%framesep = 2ex,
%rulesep = 1ex
%%framexrightmargin= 1em %
%}


% Vary the color applet  (try out your own if you like)
\colorlet{structure}{red!65!black}

%\beamertemplateshadingbackground{yellow!50}{white}


%\setbeamerfont{normal text}{family=\rmfamily}
%\setbeamerfont{frametitle}{family=\rmfamily}

% Changing the fonts: this will make the slides more readable and the math look like regular tex math
\usefonttheme{serif}



% set spaces

\setstretch{1.2}  % ÉèÖÃÐоà

\addtobeamertemplate{block begin}{\setlength\abovedisplayskip{0pt}} % reduce the large space before a block

% set section number styles



\newcommand{\secno}{Sec.\,\thesection\ }
\newcommand{\subsecno}{Sec.\,\thesubsection\ }

% set logo

 \pgfdeclareimage[width=1.0]{small-logo}{SMaLL.jpg}
%
 \logo{\vbox{\vskip0.1 \hbox{\pgfuseimage{small-logo}}}}

% set math equation fontsize

 \makeatletter
\DeclareMathSizes{\f@size}{10}{5}{5}
\makeatother

% for chinese section name
\hypersetup{CJKbookmarks=true}

\usepackage{ctex}
\begin{document}

\addtobeamertemplate{block begin}{\setlength\abovedisplayskip{0pt}}

\setbeamertemplate{itemize items}{\color{black}$\bullet$}

\pgfdeclareimage[ width=1.0cm]{small-logo}{SMaLL.jpg}
\logo{\vbox{\vskip0.1cm\hbox{\pgfuseimage{small-logo}}}}
%\title[Numerical Optimization]{Numerical Optimization}

%\date[2021]{\small    2021}



\title[数值优化]{5.7 最小二乘问题}

\bigskip

\author[]{
		 \underline{SMaLL} 
	}
	
\institute[CUP]{
		\inst{1}
		中国石油大学(华东)\\
		SMaLL 课题组   \\
		\blue{small.sem.upc.edu.cn}\\
		liangxijunsd@163.com \\ 
	 
}
		
\date[2023]{\small    2023}


\subject{5.7 最小二乘问题}
\frame{\titlepage}
	
%\frame{
%	\frametitle{Least Squares Problems}
%	\tableofcontents[hideallsubsections]
%}

\AtBeginSection[]{
\begin{frame}
	\frametitle{最小二乘问题}
	\tableofcontents[current,currentsubsection]
\end{frame}
}
		

\section{最小二乘法}

\begin{frame}
\frametitle{最小二乘法}


$$
\begin{array}{l}
	\text { minimize } \quad f(x)=\frac{1}{2}\|g(x)\|^{2}=\frac{1}{2} \sum_{i=1}^{m}\left\|g_{i}(x)\right\|^{2}\\
	\text { subject to } \quad x \in \Re^{n}
\end{array}
$$
	其中$g$ 是一个连续可微函数,具有分量函数 $g_{1}, \ldots, g_{m}$ , 其中 $g_{i}: \Re^{n} \rightarrow \Re^{r_{i}} .$
\end{frame}

\begin{frame}
\frametitle{示例}


假设模型:
$$
z=h(\theta, x).
$$

\hint{目标.  估计 $n$ 个参数 $\theta \in \bbr^{n}$. }


\begin{itemize}
	\item $h$ 是已知的能够刻画这个模型的函数
	\item $\theta \in \Re^{n}$ 是未知参数的向量
\item $x \in \Re^{p}$ 是模型的输入
	\item $z \in \Re^{r}$ 是模型的输出
	
\end{itemize}
\end{frame}

\begin{frame}
	\frametitle{示例}
	

	
	\begin{itemize}

\item\hint{数据.} $m$ 个输入-输出对 $\left(x_{1}, z_{1}\right), \ldots,\left(x_{m}, z_{m}\right)$

\item \hint{模型.}  最小化误差平方和
	$$
	\frac{1}{2} \sum_{i=1}^{m}\left\|z_{i}-h\left(\theta, x_{i}\right)\right\|^{2}
	$$
	\item E.g.,   用三次多项式近似拟合数据:
	$$
	h(\theta, x)=\theta_{3} x^{3}+\theta_{2} x^{2}+\theta_{1} x+\theta_{0}
	$$
	其中 $\theta=\left(\theta_{0}, \theta_{1}, \theta_{2}, \theta_{3}\right)$ 是未知系数的向量.
	\end{itemize}
	
\end{frame}



\section{高斯-牛顿(Gauss-Newton)法}

\begin{frame}
\frametitle{\secno\,高斯-牛顿法}

\dred{注意下面的迭代格式中, $ x^{k}$ 表示迭代点,对应最小二乘拟合问题中的参数, 并不是样本点。   }

	\begin{itemize}

\item
 \dred{给定迭代点  $x^{k}$, 高斯-牛顿迭代法的基本思想是用如下函数的线性化函数近似 $g$: }
$$
\tilde{g}\left(x, x^{k}\right)=g\left(x^{k}\right)+\nabla g\left(x^{k}\right)^{T}\left(x-x^{k}\right)
$$

\item  最小化线性化函数 $\tilde{g}$ 的范数:
$$
\begin{aligned}
	x^{k+1}=& \arg \min _{x \in \Re^{n}} \frac{1}{2}\left\|\tilde{g}\left(x, x^{k}\right)\right\|^{2} \\
	=& \arg \min _{x \in \Re^{n}} \frac{1}{2}\left\{\left\|g\left(x^{k}\right)\right\|^{2}\right.+2\left(x-x^{k}\right)^{T} \nabla g\left(x^{k}\right) g\left(x^{k}\right) \\
	&\quad \left.+\left(x-x^{k}\right)^{T} \nabla g\left(x^{k}\right) \nabla g\left(x^{k}\right)^{T}\left(x-x^{k}\right)\right\} .
\end{aligned}
$$	

\end{itemize}
	
\end{frame}

\begin{frame}
\frametitle{高斯-牛顿法}
假设 $\nabla g\left(x^{k}\right) \nabla g\left(x^{k}\right)^{T}\in \bbr^{n\times n}$ 是可逆的:
\begin{equation}	
x^{k+1}=x^{k}-\left(\nabla g\left(x^{k}\right) \nabla g\left(x^{k}\right)^{T}\right)^{-1} \nabla g\left(x^{k}\right) g\left(x^{k}\right)
\end{equation}
\begin{itemize}
\item 如果$g$ 本身是线性函数 $\Rightarrow$ $\|g(x)\|^{2}=\left\|\tilde{g}\left(x, x^{k}\right)\right\|^{2}$,  则该方法会在一次迭代后收敛。
\item 方向
$$
-\left(\nabla g\left(x^{k}\right) \nabla g\left(x^{k}\right)^{T}\right)^{-1} \nabla g\left(x^{k}\right) g\left(x^{k}\right)
$$
是下降方向因为  $\nabla g\left(x^{k}\right) g\left(x^{k}\right)  =\nabla \left( 0.5\|g(x)\|^{2}\right) |_{x = x^{k}}$ 且$\left(\nabla g\left(x^{k}\right) \nabla g\left(x^{k}\right)^{T}\right)^{-1}$ 是正定矩阵.
\end{itemize}
\end{frame}


\begin{frame}
\frametitle{高斯-牛顿法}

为了确保矩阵   \hint{$\nabla g\left(x^{k}\right) \nabla g\left(x^{k}\right)^{T}$ 是奇异矩阵(或接近奇异)时该方法也有效,迭代公式修正为:}
$$
x^{k+1}=x^{k}-\alpha^{k}\left(\nabla g\left(x^{k}\right) \nabla g\left(x^{k}\right)^{T}\dred{+\Delta^{k}}\right)^{-1} \nabla g\left(x^{k}\right) g\left(x^{k}\right)
$$
其中   $\Delta^{k}$ 是一个对角矩阵,使得:
$$
\nabla g\left(x^{k}\right) \nabla g\left(x^{k}\right)^{T}+\Delta^{k} \text { 为正定矩阵. }
$$
\begin{itemize}
    \item 高斯-牛顿法所使用的方向与梯度相关并且符合梯度下降法的收敛结果。
\end{itemize}
\end{frame}



\begin{frame}
\frametitle{与牛顿法的关系}

\begin{itemize}
  \item
假定每个 $g_{i}$ 是一个标量函数,
$$
\nabla^2 \left( 0.5\|g(x)\|^{2} \right) =
\nabla g\left(x^{k}\right) \nabla g\left(x^{k}\right)^{T}
+\sum_{i=1}^{m} \nabla^{2} g_{i}\left(x^{k}\right) g_{i}\left(x^{k}\right)
$$

\item 在高斯-牛顿法中``近似的 Hessian矩阵为'':
$$
\nabla g\left(x^{k}\right) \nabla g\left(x^{k}\right)^{T}.
$$

\item
\begin{footnotesize}
  高斯-牛顿法的迭代公式:
\begin{equation}	
x^{k+1}=x^{k}-\left(\nabla g\left(x^{k}\right) \nabla g\left(x^{k}\right)^{T}\right)^{-1}
 \nabla g\left(x^{k}\right) g\left(x^{k}\right)
\end{equation}
\end{footnotesize}


\end{itemize}

%The Gauss-Newton iterations are approximate versions of their Newton counterparts, where the second order term is neglected.
\begin{itemize}
	\item \blue{优点:}
	它简化了计算.
	\item \blue{缺点:}
	收敛速度较慢.
\end{itemize}
\end{frame}


\begin{frame}
\frametitle{与牛顿法的关系}

\begin{itemize}
  \item \hint{如果被忽略项 $\sum_{i=1}^{m} \nabla^{2} g_{i}\left(x^{k}\right) g_{i}\left(x^{k}\right)$
       在解附近很小} $\rightarrow$ 良好的收敛速度  %of the Gauss-Newton.




\item \hint{E.g.1}
当 $g$ 接近线性时, 或者当分量 $g_{i}(x)$ 在解附近很小时.

\item \hint{E.g.2}  $g(x)=0$,   $m=n$.

  $\rightarrow$  被忽略的项在解处接近零。
\item 假定 $\nabla g\left(x^{k}\right)$ 是可逆的,
$$
\left(\nabla g\left(x^{k}\right) \nabla g\left(x^{k}\right)^{T}\right)^{-1} \nabla g\left(x^{k}\right) g\left(x^{k}\right)=\left(\nabla g\left(x^{k}\right)^{T}\right)^{-1} g\left(x^{k}\right)
$$

\item 高斯-牛顿法的迭代公式化简为:
\begin{equation}
x^{k+1}=x^{k}-\left(\nabla g\left(x^{k}\right)^{T}\right)^{-1} g\left(x^{k}\right)
\end{equation}
\end{itemize}

\end{frame}




\section{增量梯度方法}

%\begin{frame}
%\frametitle{Incremental Gradient Methods}
%
% Each component $g_{i}$ in the least squares formulation is referred to as a data block. The entire function $g=\left(g_{1}, \ldots, g_{m}\right)$ is the data set.
%\end{frame}

\begin{frame}
\frametitle{\secno 增量梯度方法}

\begin{itemize}
  \item 每个分量  $g_{i}$: 一个数据块.

 %The entire function $g=\left(g_{1}, \ldots, g_{m}\right)$ is the data set.

 \item   $x^k\rightarrow x^{k+1}$: 数据块的循环

 \item \hint{初始: $\psi_{0}=x^{k}$ }

 \item
进行 $m$ 步循环:
$$
\psi_{i}=\psi_{i - 1}-\alpha^{k}
 \dred{\nabla g_{i}\left(\psi_{i-1}\right) g_{i}\left(\psi_{i-1}\right)}, \quad i=1, \ldots, m
$$

其中 $\alpha^{k}>0$ 为步长,
\dred{ 方向为第 $i$个数据块的梯度:}
$$
\left.\nabla\left(\frac{1}{2}\left\|g_{i}(x)\right\|^{2}\right)\right|_{x=\psi_{i - 1}}=\nabla g_{i}\left(\psi_{i-1}\right) g_{i}\left(\psi_{i-1}\right)
$$


\end{itemize}
\end{frame}

\begin{frame}
\frametitle{增量梯度方法}
该方法可以写作如下形式
\begin{equation}
x^{k+1}=x^{k}-\alpha^{k} \sum_{i=1}^{m} \nabla g_{i}\left(\psi_{i-1}\right) g_{i}\left(\psi_{i-1}\right)
\end{equation}

\normaltitle{不同点}

\begin{itemize}
  \item
\hint{增量梯度方法的方向: }
$$
-\sum_{i=1}^{m} \nabla g_{i}\left(\psi_{i-1}\right) g_{i}\left(\psi_{i-1}\right)
$$

\item  \hint{梯度法的方向:}
$$
-\nabla f\left(x^{k}\right)=-\sum_{i=1}^{m} \nabla g_{i}\left(x^{k}\right) g_{i}\left(x^{k}\right)
$$
\end{itemize}
\end{frame}


%\begin{frame}
%	\frametitle{Incremental Gradient Methods}
%	It can be viewed as $a$ steepest descent method with errors. We have
%	$$
%	x^{k+1}=x^{k}-\alpha^{k}\left(\nabla f\left(x_{k}\right)+e_{k}\right)
%	$$
%	where the errors $e_{k}$ are given by
%	$$
%	e_{k}=\sum_{i=1}^{m}\left(\nabla g_{i}\left(\psi_{i-1}\right) g_{i}\left(\psi_{i-1}\right)-\nabla g_{i}\left(x^{k}\right) g_{i}\left(x^{k}\right)\right)
%	$$
%\end{frame}

\begin{frame}
\frametitle{增量梯度方法}
\normaltitle{优势:}
\begin{itemize}
	\item %Estimates of $x$ become available as data is accumulated, making the approach
          适合数据流场景;

	\item %Since the loss function is not applied to all the training data, but is randomly optimized on a certain training data in each iteration,
每一轮参数的更新速度都大大加快.
\end{itemize}
\end{frame}

 
\begin{frame}
		\frametitle{练习题}
		
		
		
		\begin{enumerate}
			
			\item \hint{编程.} 生成 $m$ 个输入-输出对 $\left(x_{1}, z_{1}\right), \ldots,\left(x_{m}, z_{m}\right)$, 其中 $x_i \in \bbr$, 
			
			考虑三次多项式近似拟合数据的最小化误差平方和 模型: 
		\begin{equation}
			\min_{\theta} \ \frac{1}{2} \sum_{i=1}^{m}\left\|z_{i}-h\left(\theta, x_{i}\right)\right\|^{2}
	\end{equation}
			其中 $h$ 为三次多项式 
			$
			h(\theta, x)=\theta_{3} x^{3}+\theta_{2} x^{2}+\theta_{1} x+\theta_{0}
			$,			
			编程实现 高斯-牛顿法, 求解上述最小化误差平方和模型,计算拟合误差。
			 
		\end{enumerate}
		
\end{frame}
	
 

\end{document}
