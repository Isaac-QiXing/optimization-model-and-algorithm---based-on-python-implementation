\documentclass{beamer}
%documentclass[handout]{beamer}
%\usepackage{xeCJK}

%\usepackage[orientation=landscape,size=custom,width=16,height=12,scale=0.5,debug]{beamerposter}
\usepackage{ctex}
 % 1. packages

 % ----------- fonts and symbles ---------
\usepackage{amsmath,amssymb,amsfonts,amsthm}
%\usepackage{CJK}
\usepackage{dsfont}
\usepackage{mathrsfs}
\usepackage{eucal} % for \mathcal

%\renewcommand{\rmdefault}{ptm}


%\usepackage{fontspec}
%\newfontfamily\monaco{Monaco}

%\usepackage{mathbbold} %,bbold

 \usepackage{textcomp} % for \textnormal{\textperthousand}
% -----------------





%\usepackage{slashbox}
%\usepackage[margin=2.2cm]{geometry} % |geometry| package clash with |booktabs| package
%\usepackage{cases}
% -------- tables -------
\usepackage{booktabs} % for \toprule, \bottomrule
\usepackage{tabularx}
\usepackage{multirow}
% --------- figures ---------
\usepackage{graphicx}
% ---------- algorithms -------
\usepackage{algorithm}
\usepackage{algorithmic}
%\usepackage{footnote}
    % |footnote| package occurs error:
    % Runaway argument?
    % \def \insertfootnotetext {\@@ }\def \insertfootnotemark {\@makefnmark \ETC.

\usepackage{listings}

\usepackage[linewidth=1pt]{mdframed} % for  mdframe environment




 \usepackage{color}
 \usepackage{xcolor}     %¸ßÁÁʹÓõÄÑÕÉ«

\usepackage{setspace}
%%\usepackage{type1cm}
\usepackage{adjustbox} % for \adjustbox

\usepackage{accsupp}
\newcommand{\emptyaccsupp}[1]{\BeginAccSupp{ActualText={}}#1\EndAccSupp{}}




%%   figures and tables
\graphicspath{{figure/}}


% 2. new commands

% 2.0 common commands
%\newcommand{\bc}{\begin{center}}
%\newcommand{\ec}{\end{center}}
%\newcommand{\ba}{\begin{array}}
%\newcommand{\ea}{\end{array}}
%\newcommand{\be}{\begin{equation}}
%\newcommand{\ee}{\end{equation}}

% 2.1 colors
\definecolor{dgrey}{rgb}{0.30,0.30,0.30}
\definecolor{lred}{rgb}{0.50,0.00,0.50}
\definecolor{lblue}{rgb}{0.8,0.8,1}
\definecolor{dred}{rgb}{0.6,0,0}
\definecolor{dblue}{rgb}{0,0,0.5}
\definecolor{dgrey}{rgb}{0.35,0.35,0.35}
\definecolor{rred}{rgb}{0.9,0,0}
\definecolor{mylblue}{rgb}{0.3,0.2, 0.8}

\definecolor{commentcolor}{RGB}{85,139,78}
\definecolor{stringcolor}{RGB}{206,145,108}
\definecolor{keywordcolor}{RGB}{34,34,250}
\definecolor{backcolor}{RGB}{220,220,220}

\newcommand{\blue}[1]{{\color{blue}#1}}
\newcommand{\dblue}[1]{{\color{dblue}#1} }
\newcommand{\red}[1]{{\color{red}#1}}
\newcommand{\dred}[1]{{\color{dred}#1}}
\newcommand{\cyan}[1]{{\color{cyan}#1}}
\newcommand{\bfblue}[1]{\textbf{\color{dblue}#1} }
\newcommand{\bfred}[1]{\textbf{\color{dred}#1} }
\newcommand{\green}[1]{{\color{green}#1}}
%\newcommand{\alert}[1]{{\color{red}#1}}
\newcommand{\black}[1]{{\color{black}#1}}
\newcommand{\light}[1]{{\color{blue}\textbf{#1}}}
\newcommand{\hot}[1]{{\color{dred}#1}}
 \newcommand{\highlight}[1]{ \textbf{\color{mylblue}#1}}
 \newcommand{\important}[1]{{\color{red}#1}} % for highlighting  some words

 \newcommand{\mystar}{\dred{$^{\clubsuit}$ }}
  \newcommand{\doublestar}{\dred{$^{\clubsuit\clubsuit}$ }}

\newcommand{\mynote}[1]{{\footnotesize \color{mylblue}#1}}

 \newcommand{\hint}[1]{{\small \color{mylblue}#1}}
\newcommand{\smallhint}[1]{{\small \color{dgrey}#1}}
\newcommand{\footnotehint}[1]{{\footnotesize \color{dgrey}#1}}
\newcommand{\tinyhint}[1]{{\tiny \color{dgrey}#1}}
\newcommand{\mytitle}[1]{\medskip{\large \textbf{\color{mylblue}#1}}}
\newcommand{\normaltitle}[1]{\medskip{ \textbf{\color{mylblue}#1}}}

%\newcommand{\head}[1]{\textbf{\large\color{blue}#1}}
%\newcommand{\heading}[1]{\textbf{\large\color{blue}#1}}

\newcommand{\myfbox}[2]{ \bigskip \begin{center} \fbox{\parbox{#1}{ #2  }} \end{center}\bigskip }

\newcommand{\myvar}[1]{}
%\newcommand{\mynote}[1]{#1}

% 2.2 mathematical symbols

\newcommand{\drightarrow}{\stackrel{d.}{\rightarrow}}
\newcommand{\prightarrow}{\stackrel{p.}{\rightarrow}}
\newcommand{\bernoulli}{\textnormal{Ber}}
\newcommand{\cov}{\mathsf{Cov}}
\newcommand{\corr}{\mathbf{Corr}}
\newcommand{\regret}{\textnormal{Regret}}
\newcommand{\conv}{\textnormal{conv}}
\newcommand{\dotdiv}{\stackrel{\centerdot}{-}}
\newcommand{\dom}{\textnormal{dom}}
\newcommand{\convergenceinprob}{\stackrel{P}{\rightarrow}}
\newcommand{\convergenceindist}{\rightsquigarrow}
\newcommand{\probability}{\mathbb{P}}
\newcommand{\expectation}{\mathbb{E}}
\newcommand{\epi}{\textnormal{epi}}
\newcommand{\variance}{\mathbb{V}}
\newcommand{\var}[1]{\mathbb{V}(#1)}
\newcommand{\covariance}{\mathsf{Cov}}
\newcommand{\empiricalrisk}[1]{\hat{R}(#1)}
\newcommand{\expectedrisk}[1]{R(#1)}
\newcommand{\mgf}[1]{\psi_{#1}(\lambda)}
\newcommand{\mgfexpansion}[1]{\expectation[e^{\lambda#1}]}
\newcommand{\mgfmultivariate}[1]{\expectation[e^{\lambda^\transpose#1}]}
\newcommand{\transpose}{{\mathsf{T}}}
\newcommand{\real}{\mathbb{R}}
\newcommand{\gaussian}[2]{\mathcal{N}(#1,#2)}
\newcommand{\subGaussian}[1]{\mathsf{subG}(#1)}
\newcommand{\indicator}[1]{\mathbb{I}[#1]}
\newcommand{\x}[1]{x^{(#1)}}
\newcommand{\y}[1]{y^{(#1)}}
\newcommand{\z}[1]{z^{(#1)}}
\newcommand{\feature}{x}
\newcommand{\response}{y}
\newcommand{\supofempiricalprocess}{\|\mathbb{P}_n-\mathbb{P}\|_{\decisionspace}}
\newcommand{\decisionspace}{\mathscr{F}}
\newcommand{\decisionfunction}{f}
\newcommand{\featurespace}{\mathcal{X}}
\newcommand{\classifierestimate}{\widehat{h}}
\newcommand{\classifiertrue}{h^\star}
\newcommand{\classifier}{h}
\newcommand{\hypothesisclass}{\mathcal{H}}
\newcommand{\dataset}{\mathcal{D}}
\newcommand{\defineas}{\stackrel{\textnormal{def}}{=}}
\newcommand{\rademachercomplexity}[1]{\mathsf{Rad}_n\left(#1\right)}
\newcommand{\loss}{\ell}
\newcommand{\composite}{\circ}
\newcommand{\convexhull}{\mathsf{conv}}
\newcommand{\norm}[2][2]{\|#2\|_{#1}}
\newcommand{\shatteringcoefficient}[2]{\mathcal{S}(#1,#2)}
\newcommand{\vcdimension}[1]{\mathsf{VC}\left(#1\right)}
\newcommand{\rank}{\mathsf{rank}}
\newcommand{\innerproduct}[2]{\left\langle #1, #2\right\rangle}
\newcommand{\modelparameter}{\theta}
\newcommand{\ball}[3][]{\mathcal{B}_{{#1}}\left(#2,#3\right)}
\newcommand{\metric}{d}
\newcommand{\coveringnumber}[4][]{N_{{#1}}\left(#2,#3,#4\right)}
\newcommand{\trace}{\textnormal{tr}}
\newcommand{\std}{\textnormal{std}}
\newcommand{\sgn}{\textnormal{sign}}
%\renewcommand{\span}{\textnormal{span}}

 % do not overwrite the existing command \span
 % as it leads to an error of
 %  "Missing # Inserted in Alignment Preamble" for ``align'' environment

\newcommand{\myspan}{\textnormal{span}}

%%%
\newcommand{\rightarrowd}{\stackrel{d}{\rightarrow}}
\newcommand{\rightarrowp}{\stackrel{p}{\rightarrow}}
\newcommand{\defeq}{ \stackrel{\textnormal{def}}{=}}
\newcommand{\proj}{ \textnormal{Proj}}
\newcommand{\dist}{\textnormal{dist}}

\newcommand{\argmax}{\textnormal{argmax}}
\newcommand{\argmin}{\textnormal{argmin}}
\newcommand{\subg}{\textnormal{subG}}


 \newcommand{\bba}{\mathbb{A}}
\newcommand{\bbb}{\mathbb{B}}
\newcommand{\bbc}{\mathbb{C}}
\newcommand{\bbd}{\mathbb{D}}
\newcommand{\bbe}{\mathbb{E}}
\newcommand{\bbf}{\mathbb{F}}
\newcommand{\bbg}{\mathbb{G}}
\newcommand{\bbh}{\mathbb{H}}
\newcommand{\bbi}{\mathbb{I}}
\newcommand{\bbj}{\mathbb{J}}
\newcommand{\bbk}{\mathbb{K}}
\newcommand{\bbl}{\mathbb{L}}
\newcommand{\bbm}{\mathbb{M}}
\newcommand{\bbn}{\mathbb{N}}
\newcommand{\bbo}{\mathbb{O}}
\newcommand{\bbp}{\mathbb{P}}
\newcommand{\bbq}{\mathbb{Q}}
\newcommand{\bbr}{\mathbb{R}}
\newcommand{\bbs}{\mathbb{S}}
\newcommand{\bbt}{\mathbb{T}}
\newcommand{\bbu}{\mathbb{U}}
\newcommand{\bbv}{\mathbb{V}}
\newcommand{\bbw}{\mathbb{W}}
\newcommand{\bbx}{\mathbb{X}}
\newcommand{\bby}{\mathbb{Y}}
\newcommand{\bbz}{\mathbb{Z}}

\newcommand{\bfa}{\mathbf{a}}
\newcommand{\bfb}{\mathbf{b}}
\newcommand{\bfc}{\mathbf{c}}
\newcommand{\bfd}{\mathbf{d}}
\newcommand{\bfe}{\mathbf{e}}
\newcommand{\bff}{\mathbf{f}}
\newcommand{\bfg}{\mathbf{g}}
\newcommand{\bfh}{\mathbf{h}}
\newcommand{\bfi}{\mathbf{i}}
\newcommand{\bfj}{\mathbf{j}}
\newcommand{\bfk}{\mathbf{k}}
\newcommand{\bfl}{\mathbf{l}}
\newcommand{\bfm}{\mathbf{m}}
\newcommand{\bfn}{\mathbf{n}}
\newcommand{\bfo}{\mathbf{o}}
\newcommand{\bfp}{\mathbf{p}}
\newcommand{\bfq}{\mathbf{q}}
\newcommand{\bfr}{\mathbf{r}}
\newcommand{\bfs}{\mathbf{s}}
\newcommand{\bft}{\mathbf{t}}
\newcommand{\bfu}{\mathbf{u}}
\newcommand{\bfv}{\mathbf{v}}
\newcommand{\bfw}{\mathbf{w}}
\newcommand{\bfx}{\mathbf{x}}
\newcommand{\bfy}{\mathbf{y}}
\newcommand{\bfz}{\mathbf{z}}

\newcommand{\bfA}{\mathbf{A}}
\newcommand{\bfB}{\mathbf{B}}
\newcommand{\bfC}{\mathbf{C}}
\newcommand{\bfD}{\mathbf{D}}
\newcommand{\bfE}{\mathbf{E}}
\newcommand{\bfF}{\mathbf{F}}
\newcommand{\bfG}{\mathbf{G}}
\newcommand{\bfH}{\mathbf{H}}
\newcommand{\bfI}{\mathbf{I}}
\newcommand{\bfJ}{\mathbf{J}}
\newcommand{\bfK}{\mathbf{K}}
\newcommand{\bfL}{\mathbf{L}}
\newcommand{\bfM}{\mathbf{M}}
\newcommand{\bfN}{\mathbf{N}}
\newcommand{\bfO}{\mathbf{O}}
\newcommand{\bfP}{\mathbf{P}}
\newcommand{\bfQ}{\mathbf{Q}}
\newcommand{\bfR}{\mathbf{R}}
\newcommand{\bfS}{\mathbf{S}}
\newcommand{\bfT}{\mathbf{T}}
\newcommand{\bfU}{\mathbf{U}}
\newcommand{\bfV}{\mathbf{V}}
\newcommand{\bfW}{\mathbf{W}}
\newcommand{\bfX}{\mathbf{X}}
\newcommand{\bfY}{\mathbf{Y}}
\newcommand{\bfZ}{\mathbf{Z}}


\newcommand{\bfSigma}{\mathbf{\Sigma}}
\newcommand{\bfrho}{\mathbf{\rho}}

\newcommand{\cala}{\mathcal{A}}
\newcommand{\calb}{\mathcal{B}}
\newcommand{\calc}{\mathcal{C}}
\newcommand{\cald}{\mathcal{D}}
\newcommand{\cale}{\mathcal{E}}
\newcommand{\calf}{\mathcal{F}}
\newcommand{\calg}{\mathcal{G}}
\newcommand{\calh}{\mathcal{H}}
\newcommand{\cali}{\mathcal{I}}
\newcommand{\calj}{\mathcal{J}}
\newcommand{\calk}{\mathcal{K}}
\newcommand{\call}{\mathcal{L}}
\newcommand{\calm}{\mathcal{M}}
\newcommand{\caln}{\mathcal{N}}
\newcommand{\calo}{\mathcal{O}}
\newcommand{\calp}{\mathcal{P}}
\newcommand{\calq}{\mathcal{Q}}
\newcommand{\calr}{\mathcal{R}}
\newcommand{\cals}{\mathcal{S}}
\newcommand{\calt}{\mathcal{T}}
\newcommand{\calu}{\mathcal{U}}
\newcommand{\calv}{\mathcal{V}}
\newcommand{\calw}{\mathcal{W}}
\newcommand{\calx}{\mathcal{X}}
\newcommand{\caly}{\mathcal{Y}}
\newcommand{\calz}{\mathcal{Z}}


% 3. theorem and environments

%\newtheorem{theorem}{Theorem}%[section]
\newtheorem{proposition}{Proposition}%[section]
%\newtheorem{property}{Property}%[section]
%\newtheorem{lemma}{Lemma}%[section]
%\newtheorem{corollary}{Corollary}%[section]
%\newtheorem{definition}{Definition}%[section]
%\newtheorem{example}{Example}%[section]
%\newtheorem{remark}{Remark}%[section]
%\newtheorem{note}{Note}%[section]
%\newtheorem{problem}{Problem}%[section]
\newtheorem{exercise}{Exercise}
%\newtheorem{assumption}{Assumption}
\newtheorem*{lemma_star}{Lemma}
\newtheorem*{theorem_star}{Theorem}

%\newenvironment{summary}[1][Summary]{\par\medskip   \color{dred}\textbf{\large#1. } }{ \medskip}
%\newenvironment{remark}[1][Remark]{\par\medskip  \begin{small} \color{dblue}\textbf{#1. } }{ \end{small}\medskip}
%\renewenvironment{proof}[1][Proof]{\noindent\textbf{#1.} }{\mbox{} \hfill{\small\textrm{$\Box$}}\vspace{1ex}}
% \newenvironment{answer}[1][Answer]{\par\medskip \color{dblue}\textbf{\large#1. }}{ \medskip}

\newenvironment{summary}[1][Summary]{\par\medskip   \color{dred}\textbf{\large#1 } }{ \medskip}
\newenvironment{remark}[1][Remark]{\par\medskip   \color{dblue}\textbf{\large#1 } }{ \medskip}
\newenvironment{footnoteremark}{ \color{dblue}\begin{footnotesize} }{\end{footnotesize}}
\renewenvironment{proof}[1][Proof]{\noindent\textbf{#1.} }{\mbox{} \hfill{\small\textrm{$\Box$}}\vspace{1ex}}
 \newenvironment{question}[1][Q.]{\par\medskip {\color{lred}\large#1}}{ \medskip}
 \newenvironment{answer}[1][Answer]{\par\medskip \color{dblue}\textbf{\large#1 }}{ \medskip}

% 4. beamer setting




%\newtheorem{definition}{\textbf{¶¨Òå}}[section]
%\newtheorem{proposition}[definition] { \textbf{ÃüÌâ}}
%\newtheorem{lemma}[definition] { \textbf{ÒýÀí}}
%\newtheorem{theorem}[definition]{ \textbf{¶¨Àí}}
%\newtheorem{corollary}[definition] { \textbf{ÍÆÂÛ}}
%\newtheorem{remark}[definition] { \textbf{×¢}}
%\newtheorem{example}[definition] { \textbf{Àý}}

%\newcommand{\shadow}[1]{\begin{center}
%\bf{\textcolor{dblue}{\shadowbox{\parbox{3.8in}
% {\textcolor{red}
% {\vspace{1mm}#1}}}}}
%\end{center}}
%
%\newcommand{\head}[1]{\begin{center}
%\bf{\textcolor{dblue}{\shadowbox{\parbox{3.8in}
% {\textcolor{dred}
% {\vspace{1mm}#1}}}}}
%\end{center}}
%
%
%\newcommand{\heading}[1]{%
%  \begin{center}
%    \large\bf
%    \shadowbox{#1}%
%  \end{center}
%\vspace{1ex minus 1ex}}

% set  space above and below math equations in display style

\expandafter\def\expandafter\normalsize\expandafter{%
    \normalsize
    \setlength\abovedisplayskip{1.5ex}
    \setlength\belowdisplayskip{1.2ex}
    \setlength\abovedisplayshortskip{0.5ex}
    \setlength\belowdisplayshortskip{0.5ex}
}

% Ìí¼ÓÒ³Âë´úÂ룬¹È¸èÕÒµ½µÄ¡£
\addtobeamertemplate{navigation symbols}{}{%
    %\usebeamerfont{footline}%
    %\usebeamercolor[fg]{footline}%
    \setbeamercolor{footline}{fg=blue}
    \setbeamerfont{footline}{series=\bfseries}
    \hspace{1em}%
    \normalsize{\insertframenumber/\inserttotalframenumber}
}

% section numbering
\setbeamertemplate{section in toc}[sections numbered]
\setbeamertemplate{subsection in toc}[subsections numbered]



\lstset{                        %¸ßÁÁ´úÂëÉèÖÃ
%basicstyle=\small, % print whole listing small
%basicstyle=\footnotesize\sffamily, % print whole listing small
basicstyle=\footnotesize\rmfamily, % print whole listing small
%basicstyle=\rmfamily, % print whole listing small
    language=python,                    %PythonÓï·¨¸ßÁÁ
    %linewidth=0.9\linewidth,            %Áбílist¿í¶È
    %basicstyle=\ttfamily,              %ttÎÞ·¨ÏÔʾ¿Õ¸ñ
    commentstyle=\color{commentcolor},  %×¢ÊÍÑÕÉ«
    keywordstyle=\color{keywordcolor},  %¹Ø¼ü´ÊÑÕÉ«
    stringstyle=\color{stringcolor},    %×Ö·û´®ÑÕÉ«
    %showspaces=true,                   %ÏÔʾ¿Õ¸ñ
    numbers=left,                       %ÐÐÊýÏÔʾÔÚ×ó²à
    %numberstyle=\tiny\emptyaccsupp,     %ÐÐÊýÊý×Ö¸ñʽ
    numberstyle=\tiny,                  %ÐÐÊýÊý×Ö¸ñʽ
    numbersep=5pt,                      %Êý×Ö¼ä¸ô
    frame=single,                       %¼Ó¿ò
    framerule=0pt,                      %²»»®Ïß
    %escapeinside=@@,                    %ÌÓÒݱêÖ¾
    escapeinside=``,                    %ÌÓÒݱêÖ¾
    emptylines=1,                       %
    xleftmargin=3em,                    %list×ó±ß¾à
    backgroundcolor=\color{backcolor},  %ÁÐ±í±³¾°É«
    tabsize=4,                          %ÖƱí·û³¤¶ÈΪ4¸ö×Ö·û
    %gobble=4                            %ºöÂÔÿÐдúÂëÇ°4¸ö×Ö·û
    breaklines=true,
    extendedchars=false
    }

\lstdefinestyle{numbers}{numbers=left, stepnumber=1, numberstyle=\tiny, numbersep=10pt}
 \lstdefinestyle{nonumbers}{numbers=none}

\newcommand{\alertcode}[1]{{\color{red}#1}} % used for alerting codes

%\lstset{numbers=left, numberstyle=\tiny,
%keywordstyle=\color{blue!70},
%commentstyle=\color{red!50!green!50!blue!50},
%frame=shadowbox,
%rulesepcolor=\color{red!20!green!20!blue!20},
%escapeinside=``,
%framesep = 2ex,
%rulesep = 1ex
%%framexrightmargin= 1em %
%}


% Vary the color applet  (try out your own if you like)
\colorlet{structure}{red!65!black}

%\beamertemplateshadingbackground{yellow!50}{white}


%\setbeamerfont{normal text}{family=\rmfamily}
%\setbeamerfont{frametitle}{family=\rmfamily}

% Changing the fonts: this will make the slides more readable and the math look like regular tex math
\usefonttheme{serif}



% set spaces

\setstretch{1.2}  % ÉèÖÃÐоà

\addtobeamertemplate{block begin}{\setlength\abovedisplayskip{0pt}} % reduce the large space before a block

% set section number styles



\newcommand{\secno}{Sec.\,\thesection\ }
\newcommand{\subsecno}{Sec.\,\thesubsection\ }

% set logo

 \pgfdeclareimage[ width=1.0cm]{small-logo}{small-member/SMaLL}
%
 \logo{\vbox{\vskip0.1cm\hbox{\pgfuseimage{small-logo}}}}

% set math equation fontsize

 \makeatletter
\DeclareMathSizes{\f@size}{10}{5}{5}
\makeatother

% for chinese section name
\hypersetup{CJKbookmarks=true}

%\include{setting_2019}

\usepackage{hyperref}
\hypersetup{hidelinks,
	colorlinks=true,
	allcolors=black,
	pdfstartview=Fit,
	breaklinks=true}

\begin{document}
	%\begin{CJK*}{GBK}{kai}
	\lstdefinestyle{numbers}{numbers=left, stepnumber=1, numberstyle=\tiny, numbersep=10pt}
	\lstdefinestyle{nonumbers}{numbers=none}
	
	\addtobeamertemplate{block begin}{\setlength\abovedisplayskip{0pt}}
	
	\setbeamertemplate{itemize items}{\color{black}$\bullet$}
	
	\title[最优化模型与算法]{5.6 共轭梯度法}
	
     \bigskip
	
     \author[]{
		 \underline{SMaLL} 
	}
	
    \institute[CUP]{
		\inst{1}
		中国石油大学(华东)\\
		SMaLL 课题组   \\
		\blue{small.sem.upc.edu.cn}\\
		liangxijunsd@163.com \\ 
	 
}
		
	\date[2023]{\small    2023}
	
	
	
	\subject{ optimization}
	
	\frame{\titlepage}
	
	%\frame{
	%	\frametitle{}
	%	\tableofcontents[hideallsubsections]
	%}
	
	%\setcounter{section}{3}
	
	%\AtBeginSection[]{
	%\begin{frame}
	%	\frametitle{}
	%	\tableofcontents[currentsection,currentsubsection]
	%\end{frame}
	%} %目录	
	
	
	\frame{
		\frametitle{5.6 共轭梯度法\quad (Conjugate Gradient Method)}
		\tableofcontents[hideallsubsections]}% 显示在目录中加亮的当前章节
	
	%%
	%% 定义框架页
	%%
	\AtBeginSection[]{                              % 在每个Section 前都会加入的Frame
		%\frame<handout:1>{
		\begin{frame}
			\frametitle{5.6 共轭梯度法\quad (Conjugate Gradient Method)}
			%\tableofcontents[current,currentsubsection] % 显示在目录中加亮的当前章节
			%\tableofcontents[hideallsubsections,currentsection] % 显示在目录中加亮的当前章节
			\tableofcontents[current,hideallsubsections]
		\end{frame}
		
		%}
	}
	
	
	
	
	
	\section{共轭方向法}
	\begin{frame}
		\frametitle{\secno 共轭方向法}
		
		\normaltitle{定义}
	    \hint{给定一个$n \times n$ 的正定矩阵 $Q$,} 一个非零向量的集合 $d^1,\dots,d^k$ 是 \hint{\emph{Q共轭向量}}, 如果
		\begin{equation}
			{d^i}^{\top}Q{d^j}=0,
		\end{equation}
		对于所有 $i$ 和 $j$ 都有 $i \neq j$.
		
		
			\begin{itemize}
				\item 如果 $d^l,\dots,d^k$ 是 \emph{Q共轭向量}, 那么它们是线性无关的.
			\end{itemize}
			\normaltitle{证明}
 假设 $d^k$ 可以表示为其他向量的线性组合, $$d^k={\alpha}^1 d^1+\dots +{\alpha}^{k-1}d^{k-1},$$ 然后左乘 ${d^k}^{\top} Q$,$${d^k}^{\top}Q{d^k}={\alpha}^1{d^k}^{\top}Qd^1+
				\dots+{\alpha}^{k-1}{d^k}^{\top}Q{\alpha}^{k-1}=0,$$上式不可能成立,因为 $d^k\neq 0$ 且 $Q$ 是正定的.
			
		
		
	\end{frame}
	%==============================================================================================
	\begin{frame}
		\frametitle{\secno 共轭方向法}
		%    \vspace{5ex}
		
		\begin{itemize}
			\item \hint{Task. } 二次函数的无约束最小化
			\begin{equation}
				\min f(x)=\frac{1}{2}x^{\top}Qx-b^{\top}x.
			\end{equation}
			
			\item \hint{Inputs. }
			\begin{itemize}
				\item   给定 $n$ 个 \emph{Q} 共轭方向 $d^0,\dots, d^{n-1}$
				
				\item $x^0$ 为任意初始化向量
				
			\end{itemize}
			
			\item \dred{共轭方向迭代}
			\begin{itemize}
				\item
				\begin{equation}
					\begin{aligned}
						\alpha_k & = \dred{ \argmin_{\alpha} f(x^k+{\alpha}d^k) }\\
						& =\frac{{d^k}^{\top}(b-Qx^k)}{{d^k}^{\top}Q{d^k}} \\
					\end{aligned}
				\end{equation}
				%   $\Rightarrow$ $ =\frac{{d^k}^{\top}(b-Qx^k)}{{d^k}^{\top}Q{d^k}}$;
				
				\item
				\hint{ $
					x^{k+1}=x^k+{\alpha}^kd^k,\qquad k=0,\dots,n-1
					$ }
				
			\end{itemize}
			
		\end{itemize}
	\end{frame}
	
	%------------------------------------------
	\begin{frame}
		\frametitle{解释}
		
        \mytitle{\secno \secname}
		
		\normaltitle{定义}
			线性空间$V$的线性流形(linear manifold ): $$P=r_0+V_1=\{r_0+\alpha \mid \alpha \in V_1\},$$ 其中$V_1$是$V$的子空间,$r_0\in V$,$V_1$的维数称为线性流形$P$的维数,线性流形是直线、二维平面、\dots、$n-1$维平面的总称。
			
		
		\normaltitle{定理}
			对于Q共轭方向 $d^0,\cdots,d^{n-1}$,
			
			$\Rightarrow$  共轭方向法迭代解 $\{x_k\}$,
			满足      %\emph{successive iterates minimize $f$ over a progressively expanding linear manifold that eventually includes the global minimum of $f$}.
			%For each $k, x^{k+1}$ minimizes $f$ over the linear manifold passing through $x^0$ and spanned by the conjugate directions $d^0,\dots ,d^k$, that is,\\
			\begin{equation}
				x^{k+1}=\mathop{\arg\min}_{x \in M^k}f(x), \quad k=0,\cdots, n-1
			\end{equation}
			其中
			%\begin{equation}
			$
			%    \begin{aligned}
			M^k=\{x \mid x=x^0+v\},\quad v\in \textnormal{span}\{ d^0,\dots,d^k\}.
			$
			%   \end{aligned}
			%   \end{equation}
			
		
		
		
	\end{frame}
	%-------------------------------------------
	\begin{frame}{allowframebreaks}
		\frametitle{解释}
		\normaltitle{证明}
			
			共轭方向. $\Rightarrow$ $\forall\,i$, 
			$${\frac{d f(x^i+\alpha d^i)}{d \alpha}} \mid _{\alpha ={\alpha}^i} = {\nabla f(x^{i+1})}^{\top}d^i=0.$$
			
			我们需要展示:
			$$
			\begin{array}{cl}
				&  x^{k+1} = \argmin_{x\in M^k} f(x) \\
				\Leftrightarrow &
				(\alpha^0,\cdots,\alpha^k)
				=\argmin_{\gamma^0,\cdots,\gamma^k}\,
				f(x^0 + \gamma^0d^0 + \gamma^k d^k) \\
				\Leftrightarrow & {\frac{\partial f(x^0+{\gamma}^0 d^0+\dots+{\gamma}^k d^k)}{\partial {\gamma}^i}} \mid _{{\gamma}^j={\alpha}^j,j=0,\dots,k}=
				\langle \nabla f(x^{k+1}),d^k\rangle
				=0,\quad i = 0,\dots,k.\\
			\end{array}
			$$
			
			
			
			对于$i=0,\dots,k$,
			\begin{equation*}
				\begin{split}
					{\nabla f(x^{k+1})}^{\top}d^i&={(Qx^{k+1}-b)}^{\top}d^i\\
					&=\langle{ x^{i+1}+\sum_{j=i+1}^k {\alpha}^jd^j }, Qd^i \rangle -b^{\top}d^i\\
					&={x^{i+1}}^{\top}Qd^i-b^{\top}d^i
					={\nabla f(x^{i+1})}^{\top}d^i
					= 0.\\
				\end{split}
			\end{equation*}
			%    so,for $i=0,\dots,k$,\\
			%    \begin{equation}
			%    {\nabla f(x^{k+1})}^{\top}d^i=0,
			%    \end{equation}
			%
			% $\Rightarrow$ ${\frac{\partial f(x^0+{\gamma}^0 d^0+\dots+{\gamma}^k d^k)}{\partial {\gamma}^i}} \mid _{{\gamma}^j={\alpha}^j,j=0,\dots,k}=0,\quad i = 0,\dots,k. $
		
		
		
		
	\end{frame}
	
	
	%------------------------------------------------
	%\begin{frame}
	%    \frametitle{Interpretation}
	%    \begin{proof}
	%    which equals to $${\frac{\partial f(x^0+{\gamma}^0 d^0+\dots+{\gamma}^k d^k)}{\partial {\gamma}^i}} \mid _{{\gamma}^j={\alpha}^j,j=0,\dots,k}=0,\quad i = 0,\dots,k. $$
	%    \end{proof}
	%
	%
	%\end{frame}
	%===================================================
	\begin{frame}
		\frametitle{解释}
		\normaltitle{具体的} 当 $b = 0$,$Q = I$ (单位矩阵)时 %, the expanding manifold minimization property holds.
		\begin{itemize}
			\item  $f$ 的等值曲面是同心球;
			\item  $Q$共轭 $\longrightarrow$ 通常的正交;
			\item  $n$个正交方向的最小化  $\Rightarrow x^*$ (球的中心).
		\end{itemize}
		
		\bigskip
		
		% \begin{figure}
		%    \centering
		%    \includegraphics[width=0.35\textwidth]{conjugation1.png}
		%
		%    \end{figure}
		
		\begin{figure}
			\centering
			\includegraphics[width=0.4\textwidth]{conjugation1.png}
			\qquad
			\includegraphics[width=0.35\textwidth]{conjugation2.png}
		\end{figure}
		
	\end{frame}
	%================================================================
	\begin{frame}
		\frametitle{解释}
		
		\normaltitle{一般的}
		
		$I$ $\rightarrow $ 一般正定矩阵 $Q$  (缩放 $+$ 旋转)
		
		\dred{$y = Q^{\frac{1}{2}}x$:}  $\frac{1}{2}x^{\top}Qx = \frac{1}{2}{\parallel y \parallel}^2$.
		
		
		\begin{itemize}
			\item 如果\hint{ $w^0,\dots ,w^{n-1}$ } 是 $R^n$中任意正交非零向量,那么会有 %the algorithm
			$y^{k+1}=y^k+{\alpha}^kw^k,\quad k=0,\dots,n-1$
			
			\item 上述方程左乘 $Q^{-\frac{1}{2}}$
			
			$\rightarrow$ $x^{k+1}=x^k+{\alpha}^kd^k$, where $d^k=Q^{-\frac{1}{2}}w^k$.
			\item   \dred{方向 $d^0,\dots,d^{n-1}$是 $Q$共轭的}.
			
			
		\end{itemize}
		
		
	\end{frame}
	%========================================================
	\section{生成 $Q$共轭方向}
	\begin{frame}
		%	\frametitle{The conjugate function}
		\frametitle{\secno 生成$Q$共轭方向}
		
		\begin{itemize}
			\item \hint{Task. } 给定任意线性无关向量 ${\xi}^0,\dots,{\xi}^k$
			
			构造$Q$共轭方向 $d^0,\dots,d^k$ 对于所有的 $i =0,\dots, k$:
			\begin{equation}
				\textnormal{span}\{ d^0,\dots,d^i\}= \textnormal{span}\{  \xi^0,\dots,{\xi}^i\}.
			\end{equation}
			
			
			\item \hint{Procedure} (\emph{Gram-Schmidt procedure})
			\begin{enumerate}[(1)]
				\item $
				d^0={\xi}^0.
				$
				
				\item
				假设: 对于  $i < k$, 所选择的$Q$共轭方向 $d^0,\dots, d^i$ 可以使上述性质成立.
				
				将 $d^{i+1}$ 表示为
				\begin{equation}
					d^{i+1}={\xi}^{i+1}+  c^{(i+1),0}d^0 +\cdots + c^{(i+1),i}d^i
				\end{equation}
				
			\end{enumerate}
			
		\end{itemize}
	\end{frame}
	
	%===================================================
		\begin{frame}
		\frametitle{\secno 生成 $Q$共轭方向}
		选择 $c^{(i+1),j}$:  $d^{i+1}$是$d^0,\dots, d^i$的$Q$共轭的向量 :
		\begin{equation}
			{d^{i+1}}^{\top}Qd^j={{\xi}^{i+1}}^{\top}Qd^j+({c^{(i+1),0}d^0 +\cdots + c^{(i+1),i}d^i})^{\top}Qd^j=0.
		\end{equation}
		因为$d^0,\dots,d^i$是$Q$共轭的:
		\begin{equation}
			c^{(i+1),j}=-\frac{{{\xi}^{i+1}}^{\top}Qd^j}{{d^j}^{\top}Qd^j},\quad j=0,\dots,i.
		\end{equation}
		\begin{itemize}
			\item   ${d^i}^{\top}Qd^i\neq 0$
			\item $d^{i+1} \neq 0$
			\item  $i$ $\rightarrow$ $i + 1$,格拉姆-施密特过程的性质也显然成立
			%    \item if the vectors ${\xi}^0,\dots ,{\xi}^i$ are linearly independent, but the next vector ${\xi}^{i+1}$ is linearly dependent on these vectors, then the new vector $d^{i+1}$ will be zero
		\end{itemize}
		
	\end{frame}
	%==========================================================
	\begin{frame}
		\frametitle{步骤}
		\begin{itemize}
			\item 给定一组向量${\xi}^0,\dots,{\xi}^k$ \hint{(不一定线性无关)}
			\item 初始 $d^0={\xi}^0$
			\item 使 $d^{i+1}={\xi}^{i+1}+\sum_{m=0}^i c^{(i+1)m}d^m$,   $c^{(i+1)j}=-\frac{{{\xi}^{i+1}}^{\top}Qd^j}{{d^j}^{\top}Qd^j}$
			\item 如果$d^{i+1}=0$  $\rightarrow$ 没有将其置于$Q$共轭方向上
			\item 如果$d^{i+1}\neq 0$ $\rightarrow $ 将其置于$Q$共轭方向上
			
		\end{itemize}
		
	\end{frame}
	%====================================================
	\section{共轭梯度法}
	\begin{frame}
		\frametitle{\secno \secname}
		\begin{itemize}
			\item \dred{对梯度向量应用格拉姆-施密特过程,得到了共轭梯度法} \dred{${\xi}^0 = -{g}^0,\dots ,{\xi}^{n-1} = -{g}^{n-1}$}.
			\item $d^0=-g^0$;
			
			\item $d^k=-{g}^k+\sum_{j=0}^{k-1} d^j$, $c^{k,j}=  \frac{{g^k}^{\top}Qd^j}{{d^j}^{\top}Qd^j}$;
			\item $x^{k+1} = x^k + {\alpha}^kd^k$;
			\item 当$g^k = 0$时,该方法找到最优解.
		\end{itemize}
		
	\end{frame}
	
	
	%=====================================================
	\begin{frame}
		%\frametitle{Proof of the method terminates with an optimal solution after at most $n$ steps}
        \frametitle{命题1 }
	
         
			\dred{对于二次凸优化,CG方法在最多$n$步后得到最优解.}
		
	        \normaltitle{证明} 

		    首先用归纳法证明:迭代停止前生成的所有梯度 $g^k$是线性无关的.\\
			\begin{itemize}
				\item $g^0$本身是线性无关的,否则$g^0=0$ %, in which case the method terminates.
				\item 假设在$k$步后该方法还没有停止,那么$g^0,\dots,g^{k-1}$是线性无关的.
				\item $g^k=0$,在这种情况下迭代终止.
				\item  $g^k \neq 0$, $g^k$ 与 $d^0,\dots, d^{k-1}$正交 等价于$g^k$与$g^0,\dots, g^{k-1}$正交.
				\item $g^k$与$g^0线性无关,\dots, g^{k-1}$,这就完成了归纳法.
				
				
			\end{itemize}
			
		
	\end{frame}
	%========================================================
	\begin{frame}
		\frametitle{简化更新方程}
		
		\normaltitle{命题1}
			共轭梯度法中的方向可由以下算法生成:
       
			$d^0=-g^0$;

           \dred{ $d^k=-{g}^k+\beta^kd^{k-1}$,}
           其中$\beta^k =  \frac{{g^k}^{\top}{g^k}}{{g^{k-1}}^{\top}{g^{k-1}}} $.
			
		%\begin{itemize}
		%\item The directions of conjugate gradient method are generated by $d^k=-{g}^k+\beta^kd^j$, $\beta =  \frac{{g^k}^{\top}{g^k}}{{g^{k-1}}^{\top}{g^{k-1}}} $,$d^0=-g^0$
		%\item the method terminates with an optimal solution after at most $n$ steps
		%\end{itemize}
		
	\end{frame}
	
	
	\begin{frame}
		
			\normaltitle{证明}
			\begin{itemize}
				\item $g^{j+1}-g^j=Q(x^{j+1}-x^j)=\alpha_jQd^j$, 其中 ${\alpha}^j \neq 0$, 否则 ${g}^{j+1}={g}^j$,这意味着 ${d}^j=0$.
				\item ${g^i}^{\top}Qd^j=\frac{1}{\alpha_j}{g^i}^{\top}(g^{j+1}
				-g^j)=
				\begin{cases}
					0,\quad j=0,\dots,i-2,\\
					\frac{1}{\alpha^j}{g^i}^{\top}g^i,\quad j=i-1,
				\end{cases}$
				\item ${d^j}^{\top}Qd^j=\frac{1}{\alpha_j}
				{d^j}^{\top}(g^{j+1}-g^j)$.
				\item $\Rightarrow$  \hint{ $d^k=-g^k+\beta_kd^{k-1}, \beta_k=\frac{\ {g^k}^{\top}g^k}{{d^{k-1}}^{\top}(g^k-g^{k-1})},$ }
				\item 因为 $ d^{k-1}=-g^{k-1}+\beta_{k-1}d^{k-2}$
				$\Rightarrow$
				$\beta_k=\frac{ {g^k}^{\top}g^k}{{g^{k-1}}^\top g^{k-1}}$
				\item 根据   $\langle g^k, g^{k-1}\rangle =0$ $\Rightarrow$ $\beta_k=\frac{{g^k}^\top(g^k-g^{k-1})}
				{{g^{k-1}}^\top g^{k-1}}$.
			\end{itemize}
			
			
		
		
	\end{frame}
	
	\begin{frame}
          \frametitle{作业}
		\normaltitle{1.}
		Python编程:用共轭梯度法求解$f(x_1,x_2)=2{x_1}^2+8{x_2}^2+8x_1x_2$的最小值和最小值点。
		
	\end{frame}
	
	
	
	
	
	
	%\end{CJK*}
\end{document}
